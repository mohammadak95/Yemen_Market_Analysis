\section{Introduction}
\small  % Reduce font size for main text
\setlength{\parskip}{0.3em}  % Reduce paragraph spacing

Market integration is a cornerstone of economic resilience, particularly in regions fraught with conflict and infrastructural fragility. Yemen, enduring prolonged periods of instability, presents a unique context for examining the intricacies of commodity price transmission and market cohesion across its diverse and fragmented market landscape. While Aden and Sana’a function as critical nodes influencing price dynamics and economic stability, this study extends the analysis to encompass all major markets within Yemen. Understanding the mechanisms underpinning price transmission across these interconnected markets is imperative for formulating policies that enhance market efficiency, mitigate price volatility, and ensure food security nationwide.

This study employs advanced econometric methodologies, including Vector Error Correction Models (VECM) and spatial econometric techniques, to dissect the bidirectional nature of price transmission not only between Aden and Sana’a but also among other significant markets throughout Yemen. By analyzing a comprehensive set of commodities—ranging from staples like wheat and lentils to essential imported goods—the research delineates the speed and directionality of price adjustments, the impact of exchange rate regimes, and the spatial dependencies that characterize Yemen’s overall market structure. The presence of significant error correction coefficients in multiple transmission directions signifies a dynamic interplay among these markets, wherein price signals are reciprocally integrated, albeit at varying speeds across different commodities and regions.

Furthermore, the dual exchange rate regime in Yemen introduces a layer of complexity that affects all markets, particularly for imported commodities, by moderating price adjustment speeds and impeding seamless market integration. This study elucidates how exchange rate volatility acts as a barrier to efficient price transmission, thereby exacerbating regional price differentials and undermining market cohesion across the country. Spatial econometric analyses, quantified through Moran’s I statistics and spatial lag models, reveal a nuanced landscape of price dispersion juxtaposed with directional influences, underscoring the potential for strategic market linkages to enhance overall integration.

The motivation behind this research is to provide empirical evidence and analytical insights that can guide the unification of Yemen’s fragmented exchange rate regimes. By identifying the distortions and inefficiencies introduced by the dual exchange rates, this study aims to inform policymakers on the benefits of exchange rate stabilization and unification as pivotal steps toward achieving broader market integration. A unified exchange rate regime can streamline price signals, reduce transaction costs, and facilitate smoother arbitrage opportunities, thereby enhancing market efficiency and economic resilience.

The findings of this research hold significant implications for policymakers and international development entities, such as the World Bank, aiming to foster economic stability and resilience in conflict-affected regions. By identifying key leverage points—such as exchange rate stabilization, infrastructure enhancement, and targeted interventions in high-impact commodities—this study provides a strategic roadmap for mitigating price disparities and advancing market integration across all Yemeni markets. In doing so, it contributes to the broader discourse on economic recovery and development in post-conflict settings, offering actionable insights to support sustainable economic growth and food security in Yemen’s challenging environment.