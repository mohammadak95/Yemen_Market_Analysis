\section{Introduction}
\small  % Reduce font size for main text
\setlength{\parskip}{0.3em}  % Reduce paragraph spacing

Market integration is a cornerstone of economic resilience, particularly in regions fraught with conflict and infrastructural fragility. Yemen, enduring prolonged periods of instability, presents a unique context for examining the intricacies of commodity price transmission and market cohesion between its principal economic centers, Aden and Sana’a. These cities function as critical nodes within Yemen’s fragmented market landscape, influencing price dynamics and economic stability across diverse commodity sectors. Understanding the mechanisms underpinning price transmission between Aden and Sana’a is imperative for formulating policies that enhance market efficiency, mitigate price volatility, and ensure food security.

This study employs advanced econometric methodologies, including Vector Error Correction Models (VECM) and spatial econometric techniques, to dissect the bidirectional nature of price transmission between Aden and Sana’a. By analyzing a comprehensive set of commodities—ranging from staples like wheat and lentils to essential imported goods—the research delineates the speed and directionality of price adjustments, the impact of exchange rate regimes, and the spatial dependencies that characterize Yemen’s market structure. The presence of significant error correction coefficients in both transmission directions signifies a dynamic interplay between these markets, wherein price signals are reciprocally integrated, albeit at varying speeds across different commodities.

Furthermore, the dual exchange rate regime in Yemen introduces a layer of complexity, particularly for imported commodities, by moderating price adjustment speeds and impeding seamless market integration. This study elucidates how exchange rate volatility acts as a barrier to efficient price transmission, thereby exacerbating regional price differentials and undermining market cohesion. Spatial econometric analyses, quantified through Moran’s I statistics and spatial lag models, reveal a nuanced landscape of price dispersion juxtaposed with directional influences, underscoring the potential for strategic market linkages to enhance overall integration.

The findings of this research hold significant implications for policymakers and international development entities, such as the World Bank, aiming to foster economic stability and resilience in conflict-affected regions. By identifying key leverage points—such as exchange rate stabilization, infrastructure enhancement, and targeted interventions in high-impact commodities—this study provides a strategic roadmap for mitigating price disparities and advancing market integration. In doing so, it contributes to the broader discourse on economic recovery and development in post-conflict settings, offering empirical evidence and actionable insights to support sustainable economic growth and food security in Yemen.
