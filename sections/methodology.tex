\section{Methodology}
\small  % Reduce font size for main text

Our methodological framework integrates multiple econometric tools to provide a holistic view of market integration in Yemen. We proceed in stages: stationarity testing, cointegration analysis, VECM estimation, spatial modeling, and time-varying integration assessments.

\subsection{Stationarity and Integration Tests}
Before conducting cointegration analysis, we test whether individual time series are non-stationary and integrated of the same order. We employ both the Augmented Dickey-Fuller (ADF) test \citep{dickey1979distribution} and the Kwiatkowski-Phillips-Schmidt-Shin (KPSS) test \citep{kwiatkowski1992testing} to assess stationarity.

The ADF test takes the form:
\vspace{-0.5em}
\begin{equation}
\Delta P_t = \alpha + \beta t + \gamma P_{t-1} + \sum_{i=1}^{k} \delta_i \Delta P_{t-i} + \varepsilon_t
\end{equation}
\vspace{-0.5em}

where $\Delta P_t = P_t - P_{t-1}$ is the first-difference operator, $\alpha$ is a constant, $\beta t$ is a trend term, and $\gamma$ tests for the presence of a unit root. The optimal lag length $k$ is selected based on information criteria.

\subsection{Cointegration Analysis}
Upon confirming that series are typically integrated of order one, we employ both Engle-Granger \citep{engle1987cointegration} and Johansen \citep{johansen1988statistical} procedures to identify long-run equilibrium relationships among prices.

\subsubsection{Engle-Granger Two-Step Procedure}
First, estimate the long-run relationship:
\vspace{-0.5em}
\begin{equation}
P_{it} = \alpha_0 + \alpha_1 P_{jt} + u_t
\end{equation}
\vspace{-0.5em}

Then test the residuals for stationarity:
\vspace{-0.5em}
\begin{equation}
\Delta u_t = \theta u_{t-1} + \sum_{i=1}^{k} \phi_i \Delta u_{t-i} + \eta_t
\end{equation}
\vspace{-0.5em}

\subsubsection{Johansen Test}
For a system of prices $\mathbf{P}_t = (P_{1t}, P_{2t}, \ldots, P_{nt})'$, we consider a VAR model \citep{johansen1995}:
\vspace{-0.5em}
\begin{equation}
\mathbf{P}_t = \mu + \Phi_1 \mathbf{P}_{t-1} + \Phi_2 \mathbf{P}_{t-2} + \cdots + \Phi_p \mathbf{P}_{t-p} + \varepsilon_t
\end{equation}
\vspace{-0.5em}

This can be re-parameterized into VEC form:
\vspace{-0.5em}
\begin{equation}
\Delta \mathbf{P}_t = \Pi \mathbf{P}_{t-1} + \sum_{i=1}^{p-1} \Gamma_i \Delta \mathbf{P}_{t-i} + \mu + \varepsilon_t
\end{equation}
\vspace{-0.5em}

where $\Pi = \alpha \beta'$ contains information about long-run relationships.

\subsection{Vector Error Correction Models}
Given the presence of cointegration, VECMs capture both short-term price adjustments and the speed of reversion to the long-run equilibrium \citep{lutkepohl2005}:
\vspace{-0.5em}
\begin{equation}
\Delta X_t = \alpha \beta' X_{t-1} + \sum_{i=1}^{p-1} \Gamma_i \Delta X_{t-i} + \delta Z_t + \varepsilon_t
\end{equation}
\vspace{-0.5em}

where $\beta'$ defines the cointegrating vectors, and $\alpha$ measures the speed of adjustment.

\subsection{Spatial Econometric Models}
To incorporate geographic interdependencies, we estimate spatial lag and spatial error models \citep{anselin1988spatial, lesage2009}. We construct spatial weights matrices using K-nearest neighbor criteria:
\vspace{-0.5em}
\begin{equation}
w_{ij} = \begin{cases}
1 & \text{if } j \text{ is among } k \text{ nearest neighbors of } i \\
0 & \text{otherwise}
\end{cases}
\end{equation}
\vspace{-0.5em}

\subsubsection{Spatial Lag Model (SLM)}
\vspace{-0.5em}
\begin{equation}
y = \rho W y + X \beta + \varepsilon
\end{equation}
\vspace{-0.5em}

\subsubsection{Spatial Error Model (SEM)}
\vspace{-0.5em}
\begin{equation}
y = X \beta + u, \quad u = \lambda W u + \varepsilon
\end{equation}
\vspace{-0.5em}

where $W$ is the spatial weights matrix, and $\rho$ and $\lambda$ capture spatial dependencies.

\subsection{Time-Varying Market Integration Index}
Market integration may be dynamic, reflecting shifting frontlines of conflict, currency crises, or policy interventions. We implement a time-varying MII using a state-space model \citep{hansen2002testing}:
\vspace{-0.5em}
\begin{equation}
\begin{aligned}
y_t &= \alpha_t + \varepsilon_t, \quad \varepsilon_t \sim N(0, \sigma_\varepsilon^2) \\
\alpha_t &= \alpha_{t-1} + \eta_t, \quad \eta_t \sim N(0, \sigma_\eta^2)
\end{aligned}
\end{equation}
\vspace{-0.5em}

\subsection{Price Differential Analysis}
We analyze price differentials between markets:
\vspace{-0.5em}
\begin{equation}
\Delta P_{ijt} = \alpha + \beta_1 D_{ij} + \beta_2 C_{ijt} + \beta_3 E_t + \varepsilon_{ijt}
\end{equation}
\vspace{-0.5em}

where $D_{ij}$ measures distance, $C_{ijt}$ captures conflict intensity, and $E_t$ represents exchange rate effects.

\subsection{Diagnostic Tests}
To ensure the robustness and reliability of our econometric models, we implement comprehensive diagnostic testing following \citet{jarque1987test} and \citet{breusch1979simple}. These diagnostic tests help identify potential issues such as non-normality of residuals, heteroskedasticity, serial correlation, and spatial autocorrelation, which, if unaddressed, could compromise the validity of our model estimates and inferences.

\subsubsection{Statistical Tests}
We conduct a series of statistical tests to evaluate the underlying assumptions of our models:

\begin{itemize}
\setlength{\itemsep}{0pt}  % Reduce spacing between items
\item \textbf{Residual Normality (Jarque-Bera Test)}: The Jarque-Bera test assesses whether the residuals of our model follow a normal distribution. Normality of residuals is crucial for the validity of hypothesis tests and confidence intervals. Deviations from normality may indicate model misspecification, such as omitted variables or incorrect functional forms, and can affect the reliability of our inference.

\item \textbf{Heteroskedasticity (Breusch-Pagan Test)}: The Breusch-Pagan test examines whether the variance of the residuals is constant (homoskedasticity) or varies with the level of an explanatory variable (heteroskedasticity). Heteroskedasticity can lead to inefficient estimates and biased standard errors, which in turn affect the significance tests of our coefficients.

\item \textbf{Serial Correlation (Durbin-Watson Test)}: The Durbin-Watson test detects the presence of autocorrelation in the residuals, which occurs when residuals are correlated across time. Autocorrelation violates the assumption of independent error terms and can result from model misspecification, such as omitted lagged variables. It affects the efficiency of our estimators and the validity of hypothesis tests.

\item \textbf{Spatial Autocorrelation (Moran's I)}: Moran's I test evaluates the presence of spatial autocorrelation in the residuals, indicating whether residuals from nearby observations are correlated. Spatial autocorrelation violates the assumption of independence in spatial econometric models and suggests that spatial dependence needs to be explicitly modeled to obtain unbiased and efficient estimates.
\end{itemize}

\subsubsection{Robustness Checks}
To verify the stability and reliability of our findings, we perform a series of robustness checks:

\begin{itemize}
\setlength{\itemsep}{0pt}  % Reduce spacing between items
\item \textbf{Alternative Sample Periods}: We test the consistency of our results by analyzing different time frames. This helps ensure that our findings are not driven by specific events or anomalies within a particular period.

\item \textbf{Different Lag Specifications}: We explore various lag lengths in our time-series models to determine whether the choice of lag structure affects our results. Consistent findings across different lag specifications enhance the credibility of our conclusions.

\item \textbf{Various Spatial Weight Matrices}: To account for different notions of spatial relationships, we utilize multiple spatial weight matrices (e.g., contiguity-based, distance-based) and assess whether our spatial econometric estimates remain stable across these specifications.

\item \textbf{Multiple Threshold Specifications}: We experiment with different threshold values in our models to examine whether the identification of key relationships is sensitive to the choice of threshold. Robustness to threshold variations strengthens the generalizability of our results.
\end{itemize}
