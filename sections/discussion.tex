\section{Discussion}
\small  % Reduce font size for main text
\setlength{\parskip}{0.3em}  % Reduce paragraph spacing

The findings presented in this study highlight a complex yet navigable landscape of commodity market integration in Yemen, with Aden and Sana'a emerging as central nodes in the price transmission network. Despite persistent conflict, limited infrastructure, and a dual exchange rate regime, the core results indicate that underlying supply-demand relationships remain sufficiently robust to support meaningful cointegration. Essential staples, including wheat and lentils, demonstrate both stable long-run relationships and strong short-term adjustment capabilities. These patterns suggest a latent resilience in the national market system, and that, even in challenging environments, price signals can traverse economic and geographic boundaries.

One of the most striking outcomes is the combination of dispersion indicated by Moran's I tests with the consistently positive spatial lag coefficients. While negative Moran's I values underscore a tendency towards spatial price divergence, the strong and positive spatial lag terms confirm that neighboring markets—particularly Aden and Sana'a—do influence each other's price levels. This duality indicates that Yemen's commodity markets are not merely fragmented enclaves but rather interlinked nodes in a spatial network. Prices are not isolated; they reverberate across space, and Aden–Sana'a linkages, though tested by adversity, remain conduits of price information and stabilization.

The ECM analysis, showing significant bidirectional adjustment, reinforces the idea of anchored integration. In essence, Aden and Sana'a function as ``integration anchors''—hubs through which external shocks and policy interventions may have ripple effects, promoting alignment across a broader range of commodities and regions. While persistent price gaps exist, they are not wholly intractable. The high explanatory power of certain models underscores the influence of identifiable factors—such as transportation infrastructure, input costs, and exchange rates—on price differentials. Targeted policies can address these variables, creating an environment conducive to narrower gaps and improved food security.

The dual exchange rate regime currently operating in Yemen introduces distortions that affect price formation, particularly for imported commodities. The results reveal that while robust long-term relationships and meaningful short-term adjustments exist between key markets like Aden and Sana'a, the uneven influence of parallel exchange rates can slow down price convergence and limit the potential benefits of market integration. Indeed, commodities such as milling wheat, rice, and other internationally sourced staples are disproportionately affected by currency volatility and segmented exchange markets.

This situation means that while underlying economic fundamentals support market connectivity, exchange rate fragmentation is a critical barrier to fully realizing the potential of these linkages. The persistent price differentials in certain commodities, even when other sources of friction are accounted for, underscore that monetary policies—and especially exchange rate unification—are not peripheral concerns. Instead, they lie at the heart of achieving smoother and more efficient price transmission. In essence, current exchange rate conditions function like a filter, dampening or distorting price signals that would otherwise flow more freely between the two anchor markets.

A coherent, unified exchange rate regime emerges as a key lever for unlocking market efficiency. Current evidence suggests that aligning disparate rates could enhance responsiveness to price shocks, reduce unexplained price differentials, and streamline arbitrage opportunities. If Aden and Sana'a are to function effectively as integration anchors, they must operate within a stable, predictable monetary framework that allows prices to reflect fundamental supply-demand conditions rather than currency risk premiums.

Overall, the results challenge the narrative that ongoing conflict completely dismantles market coherence. On the contrary, the patterns revealed here provide an evidence-based foundation upon which policy practitioners, donors, and non-governmental organizations can build strategies to foster greater integration, stability, and resilience.
