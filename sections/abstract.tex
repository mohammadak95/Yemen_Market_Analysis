\begin{abstract}
\small  % Reduce font size for abstract
\noindent
This paper investigates the nature and degree of market integration and price transmission across staple commodity markets in Yemen, a country affected by protracted conflict, institutional fragmentation, and a dual exchange rate regime. Employing a rigorous econometric and spatial framework, we combine cointegration analysis, vector error correction models (VECMs), spatial econometric specifications, and time-varying market integration indices to examine long-run equilibrium relationships, short-run dynamics, and spatial dependencies in commodity price movements across all major Yemeni markets. Our findings offer granular insights into the complex interplay of conflict, market fragmentation, transaction costs, and exchange rate distortions in shaping price formation and integration patterns. Importantly, the study highlights the role of exchange rate unification in enhancing market cohesion. The results hold significant implications for policymakers, particularly in guiding the unification of exchange rates to improve market efficiency, mitigate price volatility, and ensure food security. These insights are crucial for humanitarian organizations and international development partners committed to fostering economic resilience in Yemen’s challenging environment.
\end{abstract}

\vspace{1em}
\noindent
\small
\textbf{Keywords:} Market Integration, Price Transmission, Vector Error Correction Model, Spatial Econometrics, Exchange Rate Regime, Yemen, Aden, Sana'a, Commodity Prices, Economic Resilience, Food Security

\vspace{2em}
