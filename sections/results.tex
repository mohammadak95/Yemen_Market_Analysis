\section{Results}

This section delineates the empirical findings from our comprehensive analysis of market integration and price transmission across Yemen's commodity markets, with a particular emphasis on the pivotal comparison between Aden and Sana'a. By leveraging econometric models, price differential analyses, and spatial econometric techniques, we uncover both the challenges and the underlying resilience within Yemen's fragmented markets. The results highlight significant long-run equilibrium relationships, responsive short-term dynamics, and meaningful spatial interdependencies, especially between the major trading hubs of Aden and Sana'a. These insights not only underscore the complexities induced by prolonged conflict and institutional fragmentation but also illuminate pathways for enhancing market integration and economic resilience.

\subsection{Overview of Key Findings}

Our analyses reveal that Yemeni commodity markets exhibit a blend of fragmentation and resilience. Despite the pervasive conflict, significant long-run cointegration relationships persist among key staples, indicating underlying market linkages. Price differentials between Aden and Sana'a are both statistically significant and indicative of persistent yet manageable disparities, suggesting potential avenues for improved market synchronization. Spatial econometric results further affirm that while markets are generally dispersed, there exists a positive influence of neighboring markets on each other's prices, particularly between Aden and Sana'a. These findings collectively point to a fragmented yet interconnected market structure, with Aden and Sana'a serving as crucial anchors facilitating broader market integration.

\subsection{Cointegration and Error Correction Models (ECMs)}

The Error Correction Model (ECM) results provide robust evidence of long-run equilibrium relationships and short-term price adjustment dynamics between Aden and Sana'a across various commodities. Specifically, the ECM results for Aden-to-Sana'a and Sana'a-to-Aden directions highlight the following:

\begin{itemize}
\item Beans (Kidney Red and White): Both kidney red and white beans exhibit strong cointegration between Aden and Sana'a, with adjustment coefficients ($\alpha$) of approximately 0.896 and 0.050 respectively in the South-to-North direction. The significant negative gamma coefficients indicate a robust short-term adjustment mechanism, facilitating price convergence following deviations from equilibrium. In the North-to-South direction, similar cointegration is observed, albeit with differing adjustment speeds, reinforcing the bidirectional nature of price transmission.

\item Eggs: Eggs demonstrate a substantial adjustment coefficient of 1.224 in the South-to-North ECM, reflecting a rapid return to equilibrium after price shocks. The corresponding North-to-South model also shows significant adjustment ($\alpha = -2.873$), underscoring the responsiveness of egg prices between the two markets.

\item Fuel (Diesel, Gas, Petrol-Gasoline): Fuel commodities exhibit varying degrees of price adjustment. Diesel and gas, in particular, show significant ECM coefficients, indicating effective price transmission between Aden and Sana'a. Petrol-gasoline also maintains a strong adjustment dynamic, though slightly less pronounced.

\item Lentils: Lentils display a robust cointegration relationship with an adjustment coefficient ($\alpha$) of 0.937 in the South-to-North direction and -2.708 in the North-to-South direction. This signifies a strong long-term linkage and effective short-term adjustment, facilitating price stability despite market fragmentation.

\item Milling Cost (Wheat) and Oil (Vegetable): These commodities show significant ECM coefficients, suggesting stable long-run relationships and consistent price adjustments. The dual exchange rate regime's impact is evident, particularly in the milling cost of wheat, where the adjustment speed is moderated, reflecting the complexities introduced by parallel exchange rates.

\item Onions, Potatoes, Rice (Imported), Salt, Sugar, Tomatoes, Wheat, Wheat Flour: All these commodities exhibit significant ECM coefficients in both directions, indicating persistent long-run relationships and effective price transmission mechanisms. The unified ECM results further consolidate these findings, demonstrating a general tendency towards equilibrium reversion across all commodities.
\end{itemize}

Key Insights:
\begin{itemize}
\item Bidirectional Price Transmission: The presence of significant ECM coefficients in both South-to-North and North-to-South directions for most commodities underscores the bidirectional nature of price transmission between Aden and Sana'a.

\item Speed of Adjustment: The variation in adjustment coefficients across commodities suggests differential responsiveness, with staples like wheat and lentils showing faster convergence compared to others, indicating their critical role in market stabilization.

\item Impact of Exchange Rates: The dual exchange rate regime notably affects imported commodities, moderating their price adjustment speeds and highlighting the need for exchange rate stabilization to enhance market integration.
\end{itemize}

\subsection{Price Differentials Between Aden and Sana'a}

Price differential analyses provide granular insights into the spatial disparities between Aden and Sana'a, revealing both persistent gaps and instances of convergence. The following observations emerge from the price differential results:

\begin{itemize}
\item Statistical Significance: A majority of the p-values associated with price differentials between Aden and Sana'a are exceptionally low (e.g., 2.7057204006334E-06 for kidney red beans), indicating highly significant price gaps that are unlikely due to random fluctuations.

\item Test Statistics: The negative test statistics consistently exceed the critical values at the 1\%, 5\%, and 10\% significance levels, confirming the rejection of the null hypothesis of no price differential. For instance, the test statistic for kidney red beans (-5.896) surpasses the 1\% critical value of -4.169, solidifying the presence of significant price disparities.

\item R-squared Values: The R-squared values, ranging from 0.0027 to 0.9182, reflect varying levels of explanatory power across different commodity-market pairs. High R-squared values, such as 0.9182 for potatoes in Sana'a, indicate that the model effectively explains the price differential, suggesting robust underlying market mechanisms.

\item Commodity-Specific Insights:
  \begin{itemize}
  \item Kidney Red Beans and White Beans: Persistent and significant differentials between Aden and Sana'a highlight the challenges in harmonizing prices for essential staples. However, the presence of significant coefficients suggests that these disparities can be addressed through targeted interventions.
  
  \item Eggs and Lentils: Significant price differentials for eggs and lentils between Aden and Sana'a reflect the influence of localized supply-demand imbalances and transportation constraints. Yet, the consistent responsiveness to model variables indicates potential for narrowing these gaps with improved market conditions.
  
  \item Milling Cost (Wheat): The significant differentials in milling costs point to the operational challenges within the wheat supply chain. Nevertheless, the statistical robustness of these differentials underscores the effectiveness of potential policy measures aimed at cost reduction and infrastructure enhancement.
  
  \item Onions, Potatoes, Rice (Imported), Salt, Sugar, Tomatoes, Wheat, Wheat Flour: All these commodities exhibit significant price differentials between Aden and Sana'a, affirming the fragmented market structure. However, the model's explanatory power in many cases suggests that these gaps are not insurmountable and can be mitigated through strategic market integration efforts.
  \end{itemize}
\end{itemize}

Key Insights:
\begin{itemize}
\item Persistent Yet Manageable Disparities: While significant price gaps exist between Aden and Sana'a, the statistical significance and explanatory power of the models indicate that these disparities are structured and responsive to economic variables, presenting opportunities for targeted policy interventions.

\item Potential for Convergence: The ability of the models to explain a substantial portion of price differentials, especially for high R-squared commodities, suggests that enhancing market connectivity and reducing transaction costs could facilitate price convergence between Aden and Sana'a.

\item Strategic Focus on High-Risk Commodities: Commodities with both high significance and high R-squared values, such as kidney red beans and potatoes, should be prioritized for integration efforts to maximize impact on food security and market stability.
\end{itemize}

\subsection{Spatial Dependencies and Moran's I}

Spatial econometric analyses provide a nuanced understanding of how commodity prices interact across different regions, with a special focus on the spatial linkages between Aden and Sana'a. The key findings from the spatial results are as follows:

\begin{itemize}
\item Moran's I Statistics:
  \begin{itemize}
  \item Negative Moran's I Values: Most commodities, including beans, eggs, fuel types, and staples like wheat and sugar, exhibit negative Moran's I values. This suggests a tendency towards price dispersion rather than clustering, indicating that high or low prices in one region are often surrounded by low or high prices, respectively.
  
  \item Significance: A significant proportion of Moran's I values have p-values below 0.05, confirming that these spatial patterns are statistically significant and not attributable to random chance. For instance, kidney red beans in Aden have a Moran's I of -0.0579 with a p-value of 0.001, signifying significant dispersion.
  
  \item Exceptions: Commodities such as fuel (diesel) and peas (yellow, split) in certain markets do not show significant spatial autocorrelation, indicating more random price distributions in these cases.
  \end{itemize}

\item Spatial Lag Price Coefficients:
  \begin{itemize}
  \item Positive and Significant Coefficients: All spatial lag price coefficients are positive, indicating that an increase in price in neighboring regions is associated with an increase in the base market's price. This is particularly pronounced between Aden and Sana'a, where spatial lag coefficients for key commodities like beans and wheat are highly significant.
  
  \item Implications for Market Integration: The positive spatial lag coefficients suggest that while overall spatial autocorrelation is negative, there exists a directional influence where neighboring markets influence each other's price levels. This highlights the potential for leveraging these spatial linkages to enhance market synchronization between Aden and Sana'a.
  \end{itemize}

\item R-squared Values: R-squared values from spatial models range from 0.0438 to 0.9182, indicating varying levels of model fit. High R-squared values, such as 0.9182 for potatoes in Sana'a, demonstrate strong explanatory power and effective capture of spatial dependencies.

\item Variance Inflation Factor (VIF): All VIF values are 1.0, indicating no multicollinearity issues within the spatial models. This ensures the reliability of coefficient estimates and strengthens the validity of the spatial econometric findings.
\end{itemize}

Key Insights:
\begin{itemize}
\item Directional Influence Despite Dispersion: Even though Moran's I suggests dispersion, the positive spatial lag coefficients indicate meaningful directional influence between regions, particularly between Aden and Sana'a. This duality underscores the complexity of spatial interactions in fragmented markets.

\item Role of Major Hubs in Price Transmission: Aden and Sana'a, as major market hubs, play a crucial role in transmitting price signals across regions. Enhancing the connectivity and operational efficiency between these hubs can amplify their positive influence on surrounding markets, fostering broader market integration.

\item Potential for Targeted Spatial Interventions: The significant spatial dependencies observed for key commodities between Aden and Sana'a suggest that targeted interventions aimed at strengthening these linkages—such as improving transport infrastructure or stabilizing exchange rates—can have cascading positive effects on market integration and price stability.
\end{itemize}

\subsection{Comparative Analysis: Aden vs. Sana'a}

Aden and Sana'a, as principal trading and administrative centers, exhibit distinct yet interrelated market dynamics. The comparative analysis between these two markets integrates insights from ECMs, price differentials, and spatial econometrics to elucidate their roles in facilitating market integration.

\begin{itemize}
\item ECM Insights:
  \begin{itemize}
  \item Bidirectional Adjustment: Both Aden-to-Sana'a and Sana'a-to-Aden ECMs reveal significant adjustment coefficients across multiple commodities, indicating robust bidirectional price transmission. For instance, wheat and lentils show substantial adjustment speeds in both directions, reflecting the markets' responsiveness to each other's price changes.
  
  \item Asymmetric Dynamics: While bidirectional transmission is prevalent, some commodities exhibit asymmetries in adjustment speeds. For example, the adjustment coefficient for white beans is higher in the South-to-North direction ($\alpha = 0.050$) compared to North-to-South ($\alpha = -0.116$), suggesting a slightly faster convergence from Aden to Sana'a.
  \end{itemize}

\item Price Differential Insights:
  \begin{itemize}
  \item Persistent Disparities with Convergence Potential: Significant price differentials between Aden and Sana'a for staples like white beans and wheat flour indicate persistent disparities. However, the high R-squared values (e.g., 0.5769 for white beans in Sana'a vs. Aden) suggest that these disparities are structurally driven and can be addressed through targeted market integration efforts.
  
  \item Commodity-Specific Dynamics: Commodities such as eggs and milling costs display significant price gaps but also demonstrate high explanatory power, indicating that improving factors like transportation and exchange rate stability can effectively reduce these disparities.
  \end{itemize}

\item Spatial Econometric Insights:
  \begin{itemize}
  \item Influence Across Regions: Spatial lag coefficients for commodities between Aden and Sana'a are consistently positive and highly significant, affirming that price changes in one market influence the other. This interconnectedness underscores the strategic importance of these two markets in the national price transmission network.
  
  \item Resilience Through Connectivity: The positive spatial dependencies indicate that despite the overarching dispersion, the connectivity between Aden and Sana'a remains a critical conduit for price transmission. Enhancing this connectivity can amplify positive spillover effects, fostering greater market integration and price stability.
  \end{itemize}
\end{itemize}

Key Insights:
\begin{itemize}
\item Aden and Sana'a as Integration Anchors: The bidirectional and significant price transmission between Aden and Sana'a positions them as key anchors for market integration. Strengthening their linkage can catalyze broader market harmonization across Yemen.
  
\item Strategic Focus Areas: Enhancing infrastructure, stabilizing exchange rates, and improving market information systems between Aden and Sana'a can leverage their existing robust linkages, promoting more synchronized price movements and reducing regional disparities.
  
\item Policy Leveraging: Policies aimed at reinforcing the Aden–Sana'a corridor—such as investment in transport infrastructure, currency stabilization measures, and conflict mitigation in key trade routes—can harness the inherent market linkages to drive national market integration and economic resilience.
\end{itemize}

\subsection{Resilience and Opportunities for Market Integration}

Despite the ongoing conflict and institutional fragmentation, the analysis uncovers several areas of resilience and latent integration within Yemen's commodity markets, particularly between Aden and Sana'a. These resilient patterns suggest that, with targeted interventions, significant strides can be made towards enhancing market integration and price stability.

\begin{itemize}
\item Stable Long-Run Relationships: The persistence of long-run cointegration across multiple staples indicates that fundamental supply-demand dynamics remain intact. This stability provides a foundation upon which market integration can be built, even amidst external disruptions.

\item Responsive Short-Term Dynamics: The significant short-term adjustment coefficients in ECMs highlight the markets' capacity to respond swiftly to price shocks. This responsiveness is a positive indicator of market adaptability and the potential for rapid price convergence following targeted policy measures.

\item Positive Spatial Linkages: The meaningful spatial lag coefficients between Aden and Sana'a underscore the potential for these markets to influence each other positively. By reinforcing these linkages, the overall market system can become more cohesive and less fragmented.

\item High-Explanatory Power for Key Commodities: High R-squared values for commodities like potatoes and white beans in certain market comparisons demonstrate that a substantial portion of price differentials can be explained by the models, indicating that these differentials are manageable and predictable with the right interventions.
\end{itemize}

Key Insights:
\begin{itemize}
\item Leverage Existing Strengths: The stable long-run relationships and responsive dynamics between Aden and Sana'a should be leveraged to foster broader market integration. Policies aimed at strengthening these existing strengths can have multiplicative effects across the national market system.
  
\item Targeted Interventions: Focused interventions on key commodities and strategic market corridors can enhance the overall efficiency and resilience of Yemen's commodity markets, contributing to improved food security and economic stability.
  
\item Scalable Integration Models: The successful integration between Aden and Sana'a can serve as a scalable model for other regions, facilitating a more uniform and efficient market system nationwide.
\end{itemize}

\subsection{Robustness and Consistency of Positive Signals}

To ensure the reliability and robustness of our findings, a series of robustness checks and sensitivity analyses were conducted (see Appendix Tables A4–A6). These checks involved varying model specifications, altering spatial weight matrices, and testing different lag lengths. The core positive signals identified—such as stable cointegration relationships between Aden and Sana'a, significant spatial lag coefficients, and meaningful price differentials—remained consistent across these alternative specifications. This consistency reinforces the credibility of our results and underscores the resilience of the observed market linkages.

Key Insights:
\begin{itemize}
\item Consistent Findings Across Models: The persistence of significant relationships across different model specifications indicates that the positive market integration signals between Aden and Sana'a are robust and not artifacts of specific modeling choices.
  
\item Reliability of Cointegration and Spatial Dependencies: The stable presence of cointegration and spatial dependencies in key commodities across various robustness checks highlights the underlying strength and potential for market integration between Aden and Sana'a.
  
\item Policy Confidence: The robustness of these positive findings provides policymakers and stakeholders with confidence in the identified pathways for enhancing market integration and price stability, supporting informed decision-making and strategic planning.
\end{itemize}

\subsection{Comparative Resilience: Aden vs. Sana'a}

A comparative analysis of Aden and Sana'a reveals nuanced insights into their roles as market anchors and their capacity to influence broader market integration. The following points synthesize the distinct yet complementary dynamics observed in each market:

\begin{itemize}
\item Aden:
  \begin{itemize}
  \item Market Flexibility: Aden exhibits a higher speed of price adjustment in the South-to-North ECMs for several commodities, indicating a flexible and responsive market structure capable of rapid convergence.
  
  \item Strategic Positioning: As a major port and trade hub, Aden's market dynamics are significantly influenced by international trade flows, which facilitates efficient price transmission to Sana'a.
  
  \item Infrastructure Resilience: Despite conflict-induced disruptions, the relative stability of key commodities in Aden suggests resilience in its market infrastructure, enabling sustained price transmission even under adverse conditions.
  \end{itemize}

\item Sana'a:
  \begin{itemize}
  \item Price Stability: Sana'a demonstrates substantial long-run equilibrium relationships with Aden, particularly for staples like wheat and lentils, reflecting its central role in national food security.
  
  \item Spatial Influence: The spatial lag coefficients for commodities in Sana'a indicate a strong influence on surrounding markets, reinforcing its position as a critical market hub.
  
  \item Information Dissemination: Sana'a's market is highly responsive to price changes emanating from Aden, underscoring the effectiveness of information flows and the potential for Sana'a to act as a stabilizing force in the broader market network.
  \end{itemize}
\end{itemize}

Key Insights:
\begin{itemize}
\item Complementary Roles: Aden and Sana'a function as complementary anchors within Yemen's market system, each contributing unique strengths that enhance overall market integration and resilience.
  
\item Interdependent Market Dynamics: The bidirectional and significant price transmission between Aden and Sana'a highlights the interdependence of these markets, suggesting that strengthening one can positively impact the other and, by extension, the entire market network.
  
\item Focused Integration Efforts: Targeted efforts to reinforce the Aden–Sana'a linkage can amplify their positive influence, fostering a more synchronized and efficient national market system.
\end{itemize}