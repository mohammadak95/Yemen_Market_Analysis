\section{Literature Review}
\small  % Reduce font size for main text
\setlength{\parskip}{0.3em}  % Reduce paragraph spacing

\subsection{Market Integration in Developing Countries}
Market integration is a cornerstone of economic development as it influences resource allocation, price stability, and welfare distribution \citep{fackler2001spatial, ravallion1986testing}. In integrated markets, arbitrage ensures that spatial price differentials remain within bounds defined by transport and transaction costs. Conversely, high transaction costs, infrastructural deficiencies, trade barriers, and information asymmetries can lead to persistent segmentation \citep{dercon1995market, baulch1997transfer}.

Studies in sub-Saharan Africa and South Asia illustrate that weak infrastructure, limited market connectivity, and policy-induced distortions impede price transmission and harm food security \citep{mcnew1996spatial, rashid2008spatial}. These findings resonate strongly with Yemen's context, where conflict-related disruptions, poor road infrastructure, and political fragmentation are expected to hamper the smooth functioning of commodity markets.

\citet{fackler2001spatial} present a comprehensive review of spatial price analysis, underscoring the importance of cointegration and vector error correction models (VECMs) in evaluating long-term price relationships between markets. \citet{ravallion1986testing} emphasizes the necessity of testing for market integration, particularly in the context of policy reforms. His analysis highlights how transaction costs operate as a key barrier to market integration, a notion that holds particular relevance for conflict-affected regions like Yemen \citep{mansour2021market}.

\subsection{Price Transmission in Conflict-Affected Regions}
Conflict imposes severe structural impediments to trade and information flows. Empirical evidence from Ethiopia, Somalia, and Syria underscores that ongoing conflict raises transaction costs, creates localized monopolistic structures, and erodes market integration \citep{dercon1995market, little2007unofficial, mansour2021market}. In Yemen, multiple factions control key trade routes and border crossings, restricting commerce and resulting in localized price spikes. The dual exchange rate regime—where a parallel market rate coexists with an official rate—further distorts price signals, particularly for imported staples \citep{worldbank2022yemen}.

Recent studies of Syria's conflict-affected economy \citep{mansour2021market} illustrate how control of trade routes and border crossings by local actors distorts price signals. In Yemen, a similar dynamic unfolds, as the presence of a dual exchange rate regime introduces further complexities. Exchange rate volatility influences the relative price of imports, contributing to significant price variability across regions with differing degrees of access to foreign currency \citep{worldbank2022yemen}. Such variability poses considerable challenges for price transmission, particularly in regions dependent on imports for essential commodities.

\subsection{Econometric Approaches to Market Integration}
The assessment of market integration often begins with stationarity tests and cointegration analysis to identify long-run equilibrium relationships \citep{engle1987cointegration, johansen1988statistical}. When cointegration is present, vector error correction models (VECMs) jointly characterize short-term and long-term price dynamics. More recently, spatial econometric models have been introduced to capture geographic patterns of price dependence and to measure how spatial connectivity, or the lack thereof, influences transmission \citep{anselin1988spatial, fackler2001spatial}.

The recognition that market integration can vary over time, responding to shocks, policy changes, or conflict intensity, has driven the use of time-varying integration indices and threshold cointegration techniques \citep{hansen2002testing}. These advanced methodologies allow researchers to capture the dynamic nature of market relationships and identify critical thresholds where market behavior fundamentally changes.

This methodological evolution reflects growing awareness that market integration is not a static condition but rather a dynamic process influenced by various factors including infrastructure quality, institutional capacity, and security conditions. In conflict-affected settings like Yemen, these methodological advances are particularly relevant as they can help identify how specific shocks or policy changes affect market connectivity and price transmission mechanisms.
